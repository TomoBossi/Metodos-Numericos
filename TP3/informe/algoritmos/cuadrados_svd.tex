\Function{MinimosCuadrados}{$X, y$}
    \State $U, valores\_singulares, V^T \gets \Call{SVD}{X}$
    \State $D \gets \Call{Diag}{1/valores\_singulares}$
    \State $\beta \gets {V^T}^T \cdot {D} \cdot {U}^T \cdot {y}$
    \State \Return $\beta$
\EndFunction